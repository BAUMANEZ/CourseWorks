\documentclass[12pt,a4paper]{article} 

\usepackage{fn2kursstyle}
\usepackage[russian]{babel}
\usepackage[T2A]{fontenc} 
\usepackage[utf8]{inputenc} 
\usepackage{geometry}
\usepackage{mathtools}
\usepackage{tikz}
\usepackage{pdfpages}
\usepackage[hidelinks]{hyperref}

\counterwithout{equation}{section}
\counterwithout{figure}{section}
\graphicspath{{pic/}}
\frenchspacing 

\makeatletter
\newcommand*{\rom}[1]{\expandafter\@slowromancap\romannumeral #1@}
\makeatother

\title{МАТЕМАТИЧЕСКОЕ МОДЕЛИРОВАНИЕ ТЕРМОУПРУГОГО РАЗРУШЕНИЯ ХРУПКОГО МАТЕРИАЛА}
\group{ФН2-52Б}
\author{А.\,И.~Токарев}
\supervisor{М.\,П.~Галанин}
\date{2021}

\DeclareMathOperator{\Tr}{tr}

\newcommand*\circled[1]{\tikz[baseline=(char.base)]{
            \node[shape=circle,draw,inner sep=2pt] (char) {#1};}}

\makeatletter
\newenvironment{sqcases}{%
  \matrix@check\sqcases\env@sqcases
}{%
  \endarray\right.%
}
\def\env@sqcases{%
  \let\@ifnextchar\new@ifnextchar
  \left\lbrack
  \def\arraystretch{1.2}%
  \array{@{}l@{\quad}l@{}}%
}
\makeatother

\makeatletter
\newcommand{\oset}[3][0ex]{%
  \mathrel{\mathop{#3}\limits^{
    \vbox to#1{\kern-2\ex@
    \hbox{$\scriptstyle#2$}\vss}}}}
\makeatother

\begin{document}
    \maketitle
    \tableofcontents
    \pagebreak

    \section-{Введение}
    
    \section{Постановка задачи}

    \section-{Заключение}

    \begin{thebibliography}{9}
  
      \bibitem{Karelia} Тензоры напряжений и деформаций. URL: \url{http://solidstate.karelia.ru/p/tutorial/ftt/Part4/part4_1.htm}

      \bibitem{Deform} Тензор деформаций. SolverBook - онлайн сервисы для учебы. URL: \url{http://ru.solverbook.com/spravochnik/fizika/tenzor-deformacii/}

      \bibitem{Flex} Теория упругости. Wikipedia –- свободная энциклопедия. URL: \url{https://en.wikipedia.org/wiki/Linear_elasticity}

      \bibitem{Galanin} Галанин М.П. Методы численного анализа математических моделей/М.П. Галанин, Е.Б. Савенков.–М. : Изд-во МГТУ им. Н.Э. Баумана, 2010.–591, [1] с.: ил. (Математическое моделирование в технике и технологии)

      \bibitem{GalaninAndOthers} Математическое моделирование разрушения хрупкого материа- ла под действием тепловых нагрузок / М.П. Галанин [и др.] // Препринты ИПМ им. М.В. Келдыша. 2013.No 100. 36 с. URL: \url{http://library.keldysh.ru/preprint.asp?id=2013-100}

    \end{thebibliography}

    \end{document}