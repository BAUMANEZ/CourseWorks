\documentclass[12pt,a4paper]{article} 

\usepackage{fn2kursstyle}
\usepackage[russian]{babel}
\usepackage[T2A]{fontenc} 
\usepackage[utf8]{inputenc} 
\usepackage{geometry}
\usepackage{mathtools}
\usepackage{tikz}
\usepackage{pdfpages}
\usepackage[hidelinks]{hyperref}

\counterwithout{equation}{section}
\counterwithout{figure}{section}
\graphicspath{{pic/}}
\frenchspacing 

\title{КУСОЧНО-ПАРАБОЛИЧЕСКИЙ МЕТОД НА ЛОКАЛЬНОМ ШАБЛОНЕ ДЛЯ ЗАДАЧ ГАЗОВОЙ ДИНАМИКИ}
\group{ФН2-62Б}
\author{А.\,И.~Токарев}
\supervisor{В.\,В.~Лукин}
\date{2022}

\begin{document}
    \maketitle
    \tableofcontents
    \pagebreak

    \section-{Введение}
    Одним из наиболее удачных вычислительных методов решения гиперболических уравнений является кусочно-параболический метод (с англ. Piecewise-Parabolic Method, PPM), разработанный для моделирования течения жидкостей и газов и применяемый в астрофизике. Он обладает порядком аппроксимации $ O(\tau^2 + h^3) $. Несмотря на великолепную точность, данный метод имеет ряд недостатков: концы парабол на разностных ячейках связываются путем реконструкции переменных на расширенном четырехточечном шаблоне, что повышает диссипацию в схеме. Кроме того, PPM дает достаточно точный результат на гадких решениях, а вот на разрывах происходят ощутимые осцилляции.

    Целью данной курсовой работы является анализ улучшенного метода PPM -- кусочно-параболический метод на локальном шаблоне (PPML). Его основное отличие заключается в том, что граничные точки парабол внутри разностых ячеек определяются с предыдущего временного слоя по методу характеристик, что позволяет точно описывать разрывные решения и избегать накопления лишней диссипации.
    
    В качестве анализа будет приведено сравнение точности методов PPM и PPML на примерах одномерных задач. Также проведем демонстрацию рассматриваемого метода на нескольких двумерных задач газовой динамики.

    \newpage
    
    \section{Постановка задачи}

    \subsection{Кусочно-параболический метод. PPM}

    \section-{Заключение}

    \end{document}